\documentclass[a4paper,10pt]{article}
\usepackage[utf8x]{inputenc}
\usepackage{amsmath}
%\usepackage{abssymb}
%opening
\title{Thesis Draft 1}
\author{Anupam Prasad Vedurmudi}

\begin{document}

\maketitle

% \begin{abstract}
% 
% \end{abstract}

\section{Scattering in 1D}
\subsection{Preliminaries}
We define the Hamiltonian operator as,
\begin{equation}
 H = H_0+V\\
\end{equation}
\noindent Where, $H_0$ is the free Hamiltonian given by, 

\begin{equation}
 H_0=-\frac{\Delta}{2}
\end{equation}
\noindent $V$ is the potential.

\noindent The Green's function is formally defined as,
\begin{eqnarray}\label{GFdefinition}
 G\left(E+i\epsilon\right) &=\left(E+i\epsilon-H\right)^{-1}\\
 G_0\left(E+i\epsilon\right) &=\left(E+i\epsilon-H_0\right)^{-1}
\end{eqnarray}

\noindent The Lippmann-Schwinger equation for the Green's function is given by,
\begin{equation}\label{LS1}
 G=G_0+G_0VG
\end{equation}

\noindent Subsequently, the Born series for $G$ is obtained by recursively substituting
the $(n-1)^{th}$ order approximation for $G$ in the R.H.S of \eqref{LS1} to get the $n^{th}$
order term. The $0^{th}$ order approximation is the free Green's function $G_0$, the $1^{st}$
order approximation (the Born approximation) is given by,

\begin{equation}\label{BornApproximation}
 G=G_0+G_0VG_0
\end{equation}

The Born series can thus be written as,

\begin{equation}\label{BornSeriesExpanded}
 G=G_0+G_0VG_0+G_0VG_0VG_0+\ldots
\end{equation}

\noindent Or more compactly,
\begin{equation}\label{BornSeriesCompact}
 G=\displaystyle\sum_{n=0}^{\infty}G_0\left(VG_0\right)^n
\end{equation}


\subsection{Modification of the Hamiltonian}
In this section we use the techniques outlined in \cite{KukPom76}. It this paper, the Born
series is rearranged by using orthogonally projecting pseudopotentials (OPP). It is then
proved that the rearranged Born series converges for all negative and small positive energies
even if the system contains bound states.

Let the bound states (indexed by $a$) of the Hamiltonian $H$ be denoted by $|\psi_a\rangle$.
The potential is modified by the operator $\Gamma$ given by,
\begin{equation}
  \Gamma=\lambda\displaystyle\sum_a |\psi_a\rangle\langle\psi_a|
\end{equation}
\noindent The rate of convergence of the Born series is controlled by the parameter $\lambda$.

The modified Hamiltonian and the modified free Hamiltonian are denoted by,

\begin{eqnarray}
 \widetilde{H}&=&H+\Gamma\\
 \widetilde{H}_0&=&H_0+\Gamma
\end{eqnarray}

The modified Greens functions $\widetilde{G}$ and $\widetilde{G}_0$ are defined similarly as
in \eqref{GFdefinition}. The modified Born Series for $\widetilde{G}$ is given by,

\begin{equation}\label{BornSeriesModCompact}
 \widetilde{G}=\displaystyle\sum_{n=0}^{\infty}\widetilde{G}_0\left(V\widetilde{G}_0\right)^n
\end{equation}

\section{Finite Element Method}
The Finite Element Method (FEM) is a numerical technique used to solve differential equations.
The method exploits the locality of the differential operator. The solution space is divided into
subspaces. These subspaces take the form of a mesh, i.e., a collection of simplices - triangles
in 2D, tetrahedra in 3D, etc. The function is then approximated using smooth basis functions with support in these subspaces.
These basis functions with finite support are referred to as finite elements.

Let $\psi$ be a solution to a differential equation. We denote the basis functions by $\phi_n$,
\begin{equation}\label{femfirst}
 \psi(\mathbf{x})\simeq\widetilde{\psi}(\mathbf{x})=\displaystyle\sum_n^N c_n\phi_n(\mathbf{x})
\end{equation}

A feature of FEM is that the basis functions have a finite overlap i.e., they are not orthogonal.
The overlap matrix in general has a block-diagonal form. The overlap matrix is denoted by,
\begin{equation}\label{overlap}
 S_{nm}=\int d^3x\phi_n(\mathbf{x})\phi_m(\mathbf{x})
\end{equation}

We also define the matrix,
\begin{equation}\label{diff_overlap}
 B_{nm}=\int d^3x\nabla\phi_n(\mathbf{x}).\nabla\phi_m(\mathbf{x})
\end{equation}

\subsection{Boundary Conditions}
In order to solve the differential equations we must also specify appropriate boundary
conditions. Let $\Omega$ be the solution space. We denote its boundary by $\partial\Omega$.
\section{FEM Formulation of Quantum Mechanics}
The Schrödinger equation for a wave function $\psi$ is given by,

\begin{equation}\label{schroedinger1}
 -\frac{\Delta}{2}\psi+V(\mathbf{x})\psi=E\psi
\end{equation}

\noindent Using the FEM approximation defined in \eqref{femfirst},
\begin{equation}\label{schroedinger}
 -\frac{\Delta}{2}\displaystyle\sum_n^N c_n\phi_n(\mathbf{x})+\displaystyle\sum_n^N c_n V(\mathbf{x})\phi_n(\mathbf{x})=E\displaystyle\sum_n^N c_n\phi_n(\mathbf{x})
\end{equation}
\noindent We get the weak formulation of the problem by multiplying both sides by $\phi_m(\mathbf{x})$ and integrating over. After integrating by parts, the kinetic term gives us
the following matrix element,
\begin{equation}\label{kineticterm}
 \int d^3x\phi_m(\mathbf{x})\Delta\phi_n(\mathbf{x}) = -\int d^3x\nabla\phi_m(\mathbf{x}).\nabla\phi_n(\mathbf{x}) = -B_{mn}
\end{equation}

\noindent The potential term is calculated from the known functions $\phi_n$ and is defined as,
\begin{equation}\label{potentialterm}
 \int d^3xV(\mathbf{x})\phi_m(\mathbf{x})\phi_n(\mathbf{x}) = V_{mn}
\end{equation}

\noindent Finally, the R.H.S is given by,
\begin{equation}
 E\int d^3x\phi_m(\mathbf{x})\phi_n(\mathbf{x})=ES_{mn}
\end{equation}

\noindent We define the coefficient vector $\mathbf{\underline{c}}=[c_1,c_2,\ldots,c_N]^T$. The Schrödinger equation reduces to 
the following eigenvalue problem,
\begin{equation}\label{schroedingerfem}
  H\mathbf{\underline{c}}=ES\mathbf{\underline{c}}
\end{equation}
\noindent Where, $H=\left(\frac{B}{2}+V\right)$

The inner product of two wavefunctions $\phi$ and $\chi$ is given by,
\begin{eqnarray}\label{innerproduct}
 \int d^3x \chi^*(\mathbf{x})\psi(\mathbf{x}) &=& \displaystyle\sum_{m,n}^{N}c_md_n\int d^3x \phi_m(\mathbf{x})\phi_n(\mathbf{x})\\
 &=& \mathbf{c}^{\dagger}S\mathbf{d}
\end{eqnarray}
\noindent Where, $\mathbf{c}^{\dagger}=(\mathbf{c}^*)^T$.

\subsection{Momentum Eigenstates}
We now want to find an FEM representation for free eigenstate of momentum $\mathbf{k}$. 
It is most convenient to do so by defining a ``cosine eigenvector'',
\begin{equation}\label{cosevec}
 |\mathbf{k}\rangle = exp(-i\mathbf{k}.\mathbf{x}) + exp(i\mathbf{k}.\mathbf{x})
\end{equation}
These can be obtained by ``solving'' the free Schrödinger equation on a grid with open
boundaries, i.e., by solving the eigenvalue problem,

\begin{equation}\label{freeschroedinger}
 H_0\psi=\lambda S\psi
\end{equation}
for $\psi$.

$|\mathbf{k}\rangle$ has the following properties,
\begin{eqnarray}
 \langle\mathbf{k}|S|\mathbf{k}\rangle &=&\int_{\Omega} d^3x\\
 \langle\mathbf{k}^{\prime}|S|\mathbf{k}\rangle &\simeq& 0
\end{eqnarray}
Here, $\int_{\Omega} d^3x$ is the volume of the solution space.

\subsection{Born Series - FEM Formulation}
We calculate the FEM representation of the Greens Function using \eqref{schroedingerfem}. We require that,
\begin{equation}\label{greenfemreq}
 G(z)\psi=(z-E_0)^{-1}S\psi
\end{equation}
\noindent Where, $\psi$ is an eigenstate of $H$ of energy $E_0$. Thus, we define the Green's function as,
\begin{equation}\label{greenfemdef}
 G(z)=\left(z\mathbf{I}-HS^{-1}\right)^{-1}S
\end{equation}
\noindent Where, $\mathbf{I}$ is the identity matrix. The free Greens function is defined similarly by setting
$H=B/2$. We will ultimately be testing the convergence of, $\langle\mathbf{k}|G(z)|\mathbf{k}\rangle$.






\bibliographystyle{plain}
\bibliography{draft1.bib}
\end{document}
